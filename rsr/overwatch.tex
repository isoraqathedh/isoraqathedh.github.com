\documentclass{article}
\usepackage[utf8]{inputenc}
\usepackage[T1]{fontenc}

\begin{document}
% Title block
\title{Road sign report – Overwatch}
\date{2016-06-17}
\author{Isoraķatheð Zorethan}
\maketitle
% End title block

\section{Introduction}
\textit{Overwatch} is a video game created by Blizzard Studios
and released in 2016.
It's a game where the player picks up a character, termed a hero,
and simply beat each other silly until someone completes an objective.
Being set in the not-so-distant future, it contains a couple of traffic signs.
It is this admittedly trifling detail that this article will discuss today,
specifically its various nations' compliance with the modern-day
Vienna Convention or if appropriate the American MUTCD,
as well as fidelity to the settings of the world in two other levels:
traffic chirality, left- or right-hand-side traffic;
and unit choice, either metric or imperial.
In short, it has a very satisfactory compliance with both,
and has maintained an excellent attention
of whether the traffic drives on the left or the right.

The purpose of this article
is not to criticise the game for its faults on road design,
but to entertain and compile information
on the design and meaning of traffic control devices.
It is undoubtedly a vanishingly small part of the game,
one that virtually vanishes in the minds of most,
so the author hopes that any such criticism of the game
can be taken less seriously.

Additionally, as the author has not had the opportunity to buy the game,
any supporting visuals would be screenshots from YouTube.
Proper attribution will be given to those screenshots in the form of a caption.

\section{Route 66}
This sis set in the analogue of the United States,
and this means that we take reference to the United State's MUTCD.

% Font issues (series D guys, not this arial stuff)
% Units: correct
% design: mostly correct
% drive on side: right, correct.

\section{King's Row}
% Font: n/a
% units: n/a
% design: incorrect arrows
% drive on side: left, correct
King's Row is set in the United Kingdom, and there the rules follow the TSRGD.
Encouragingly, the cars there appear to drive on the left,
which is more than what is usually expected of video games.
In fact, what little signage present are mostly as expected,
with two slightly disappointing deviations from the British standard:
the first are the shape of the arrows, which are inconsistent
and resembles Belgian and French signs,
and the second is that the roundabout direction signs are ``keep left'',
when they should be ``turn left''.
Overall, it is valiant effort that unfortunately does not show
because of obvious flaws in placement and design.

\end{document}

%%% Local Variables:
%%% mode: latex
%%% TeX-master: t
%%% End:

%  LocalWords:  Overwatch
